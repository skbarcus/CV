% LaTeX Curriculum Vitae Template
%
% Copyright (C) 2004-2009 Jason Blevins <jrblevin@sdf.lonestar.org>
% http://jblevins.org/projects/cv-template/
%
% You may use use this document as a template to create your own CV
% and you may redistribute the source code freely. No attribution is
% required in any resulting documents. I do ask that you please leave
% this notice and the above URL in the source code if you choose to
% redistribute this file.
%
% If you have never used LaTeX before, talk you your advisor about
% how to compile this into a PDF.

\documentclass[letterpaper,10pt]{article}

\usepackage{hyperref}
\usepackage[hmargin=1.5cm,vmargin=1.5cm]{geometry}
% Comment the following lines to use the default Computer Modern font
% instead of the Palatino font provided by the mathpazo package.
% Remove the 'osf' bit if you don't like the old style figures.
\usepackage[T1]{fontenc}
\usepackage[sc,osf]{mathpazo}

\usepackage{setspace}

\usepackage{mdwlist}

\usepackage{amsmath}%Lets you use \text{} in equations.

% Set your name here
\def\name{Scott K. Barcus}

% Replace this with a link to your CV if you like, or set it empty
% (as in \def\footerlink{}) to remove the link in the footer:
\def\footerlink{}

% The following metadata will show up in the PDF properties
\hypersetup{
  colorlinks = true,
  urlcolor = black,
  pdfauthor = {\name},
  pdfkeywords = {CV, physics},
  pdftitle = {\name: Curriculum Vitae},
  pdfsubject = {Curriculum Vitae},
  pdfpagemode = UseNone
}

\geometry{
  body={6.5in, 8.5in},
  left=1.0in,
  top=1.0in
}

% Customize page headers
\pagestyle{myheadings}
\markright{\name}
\thispagestyle{empty}

\usepackage{titlesec}

% Custom section fonts
\usepackage{sectsty}
\sectionfont{\rmfamily\mdseries\Large}
\subsectionfont{\rmfamily\mdseries\large}
%\subsectionfont{\rmfamily\mdseries\itshape\large}

% Other possible font commands include:
% \ttfamily for teletype,
% \sffamily for sans serif,
% \bfseries for bold,
% \scshape for small caps,
% \normalsize, \large, \Large, \LARGE sizes.

% Don't indent paragraphs.
\setlength\parindent{0em}

% Make lists without bullets
\renewenvironment{itemize}{
  \begin{list}{}{
    \setlength{\leftmargin}{1.5em}
  }
}{
  \end{list}
}

\begin{document}
\renewcommand{\baselinestretch}{0.0}{
% Place name at left
{\huge \name}

% Alternatively, print name centered and bold:
%\centerline{\huge \bf \name}

\vspace{0.25in}

\begin{minipage}{0.45\linewidth}
  \href{http://www.wm.edu/}{101 Lake Powell Road Apartment L} \\
  Williamsburg, VA 23185 \\
\end{minipage}
\begin{minipage}{0.45\linewidth}
  \begin{tabular}{ll}
    Phone: & (651) 324-9205 \\
    Email: & \href{mailto:me@email.wm.edu}{\tt skbarcus@jlab.org} \\
  \end{tabular}
\end{minipage}

%\titlespacing{\section}{0pt}{\parskip}{-\parskip}
%\titlespacing\section*{0pt}{12pt plus 4pt minus 2pt}{0pt plus 2pt minus 2pt}
%{0pt}{5.5ex plus 1ex minus .2ex}{4.3ex plus .2ex}

%\setlength{\parskip}{\baselineskip}
%\newlength{\aftersection}
%\setlength{\aftersection}{0pt}
%\addtolength{\aftersection}{0pt}%-1\parskip}
%\newlength{\beforesection}
%\setlength{\beforesection}{0pt}
%\addtolength{\beforesection}{0pt}%-1\parskip}
%\titlespacing*{\section}{-50pt}{\aftersection}{\beforesection}

%\section*{Education}

{\Large{\bf{Education}}}
\begin{itemize}
	\item {\large{\bf{Graduate}}}
	\item {\large{\textit{The College of William \& Mary}}}
		\begin{itemize}\itemsep1pt \parskip2pt \parsep0pt
 	 		\item $\bullet$ Ph.D. Physics, March 18$^{\text{th}}$ 2019.
 	 			\begin{itemize}
 	 				\item - \textit{Extraction and Parametrization of Isobaric Trinucleon Elastic Cross Sections and Form Factors\footnote{Download link: \url{https://misportal.jlab.org/ul/publications/view_pub.cfm?pub_id=16118}.}}
 	 			\end{itemize}
  			\item $\bullet$ M.S. Physics, 2014.
		\end{itemize}
	\item {\large{\bf{Undergraduate}}}
	\item {\large{\textit{Drake University}}}
		\begin{itemize}
			\item $\bullet$ B.S. Physics, 2012.
			\item $\bullet$ B.S. Astronomy, 2012.
			\item $\bullet$ B.S. Mathematics, 2012.
		\end{itemize}
\end{itemize}

%\clearpage

%\section*{Research Experience}
{\Large{Research Experience}}
\begin{itemize}\itemsep5pt \parskip0pt \parsep0pt

\item {\large {\bf Jefferson Laboratory Postdoctoral Research April 2019-Present} }
	\begin{itemize}\itemsep5pt \parskip0pt \parsep0pt
	
		\item \textbf{Hardware:}
			\begin{itemize}\itemsep5pt
				\item $\bullet$ Commissioning the Hadron Calorimeter (HCal) for upcoming SBS experiments:
					\begin{itemize}\itemsep2pt
						\item - Calibrated and maintained the PMTs.
						\item - Co-designed and physically implemented the cable maps. 
						\item - Installed, tested, and calibrated the electronics repairing them when necessary.
						\item - Used HCal hardware to measure cosmic ray data in preparation for executing the HCal commissioning plan when installed in Hall A for the SBS program. 
					\end{itemize}
				\item $\bullet$ Electronics testing and cable making for the BigBite spectrometer detector package.
				\item $\bullet$ Designed plan to stabilize BigBite hodoscope PMTs so they can bear their weight without snapping off of their blocks. 
			\end{itemize}
			
		\item \textbf{Data Acquisition:}
			\begin{itemize}\itemsep5pt
				\item $\bullet$ Commissioned HCal DAQ System:
					\begin{itemize}\itemsep2pt
						\item - Implemented software and hardware for 18 flash ADC and 5 F1TDC modules to measure particle energies and timings using the Hall A analyzer.
						\item - Used fADC and F1TDC modules to measure cosmic rays for HCal commissioning tests including electronics tuning and timing resolution experiments. 
					\end{itemize}
				\item $\bullet$ Wrote funding proposal to use machine learning trained neural networks for particle ID on the HCal. This algorithm will be loaded onto FPGAs and used as a fast trigger to suppress background events.
				\item $\bullet$ Updated Hall A counting house DAQ scripts for PREX/CREX running.
			\end{itemize}
			
		\item \textbf{Spokesperson for Hall C Double-Polarization Asymmetry Experiment E12-06-121A:}
			\begin{itemize}\itemsep5pt
				\item $\bullet$ Wrote proposal\footnote{Proposal link: \url{https://hallcweb.jlab.org/wiki/images/b/bc/PAC_Proposal_3He_Polarization_Observables.pdf}} for run group addition to the d$_2^n$ experiment for the first ever measurement of the $^3$He electromagnetic form factors using polarization observables at high $Q^2$.
				\item $\bullet$ Presented the proposal to the $A_1^n$/$d_2^n$ collaboration meeting, and incorporated the collaboration's input gaining their support.
				\item $\bullet$ Planned and wrote PAC defense for the proposal which was presented to the PAC by $d_2^n$ spokesperson Brad Sawatzky.
				\item $\bullet$ PAC approved the proposed experiment, and it is scheduled to run in Spring 2020.
			\end{itemize}
			
		\item \textbf{Physics Analyses:}
			\begin{itemize}\itemsep5pt
				\item $\bullet$ Wrote software to analyze timing resolution of the HCal using fADC/TDC cosmic ray data.
				\item $\bullet$ Created HCal event displays for offline analysis. Work is ongoing for online analysis displays.
				\item $\bullet$ Calculated asymmetry rates for future $^3$He double-polarization asymmetry experiment in Hall C. 
				\item $\bullet$ Created Monte Carlo simulation to calculate uncertainties for the Hall C polarized $^3$He experiment. 
				\item $\bullet$ Calculated improved $^3$H and $^3$He form factor fits from world data with sum of Gaussians technique.
				\item $\bullet$ Wrote software to quantify the uncertainty introduced to $^3$H and $^3$He charge radii based on the common practice of renormalizing data such that the charge form factor be unity at $Q^2=0$.
				\item $\bullet$ Used robust regression analysis to show how analytic choices, like model selection, greatly influence physics results (e.g. proton charge radius) using the Initial State Radiation cross section data.
			\end{itemize}
			
		\item \textbf{Communication:}
			\begin{itemize}\itemsep5pt
				\item $\bullet$ Presented regular updates on the HCal at weekly collaboration meetings.
				\item $\bullet$ Presented $^3$He double-polarization asymmetry experiment at DNP meeting in Arlington, Virginia.
				\item $\bullet$ Co-authored paper\footnote{ArXiv link: \url{https://arxiv.org/abs/1902.08185}} explaining the importance of analytic choices and their impact on physics results for Physical Review C (in the late stages of review).
				\item $\bullet$ Mentored numerous graduate/undergraduate students as they learned to be researchers at JLab.
				\item $\bullet$ Contact person for HCal installation in Hall A. Responsible for coordinating between technical staff and the SBS collaboration.
			\end{itemize}
			
	\end{itemize}

\item {\large {\bf Jefferson Laboratory Elastic $^{3}$He Cross Section Extraction 2017-2019} }
	\begin{itemize}\itemsep5pt \parskip0pt \parsep0pt
		\item \textbf{Cross Section Extraction:}
			\begin{itemize}\itemsep5pt
				\item $\bullet$ Calculated yield of electrons elastically scattered off gaseous $^{3}$He target during experiment E08-014.   
				\item $\bullet$ Performed background subtraction of the target cell's aluminium walls identifying events scattered from $^{3}$He.
				\item $\bullet$ Determined the efficiencies of detectors including the VDCs, gas Cherenkov, and pion rejectors.
				\item $\bullet$ Calculated the cross section for elastic scattering off of $^{3}$He in the high Q$^2$ region of Q$^2 \approx 34.2$ fm$^{-2}$.
			\end{itemize}
		\item \textbf{Simulation:}
			\begin{itemize}\itemsep5pt
				\item $\bullet$ Performed Monte Carlo simulations of elastic electron scattering off $^{3}$He and modelled charged particle transport through the spectrometers and detectors. %with JLab's SIMC package. 
				\item $\bullet$ Used Monte Carlo simulation results to determine the acceptance for elastically scattered electrons of the Hall A high resolution spectrometers during experiment E08-014. 
				\item $\bullet$ Modelled the radiative effects present in experimental data using Monte Carlo simulation to apply radiative corrections to the $^{3}$He cross section.
			\end{itemize}
		\item \textbf{Global Fits and Form Factors:}
			\begin{itemize}\itemsep5pt
				\item $\bullet$ Compiled a database containing the world data on $^{3}$H, $^{3}$He, and $^{4}$He cross sections and form factors.
				\item $\bullet$ Developed ROOT (C++) code to fit $^{3}$H and $^{3}$He cross section world data with a sum of Gaussians (SOG) technique. 
				\item $\bullet$ Extracted charge and magnetic form factors from these global fits incorporating modern data. 
				\item $\bullet$ Calculated charge densities for $^{3}$H and $^{3}$He from new fits and extracted charge radii. 
			\end{itemize}
			
			\item \textbf{Communication:}
			\begin{itemize}\itemsep5pt
				\item $\bullet$ Presented results at the Gordon Research Conference in August 2018.
				\item $\bullet$ Presented results at the APS Division of Nuclear Physics meeting in October 2018.
			\end{itemize}
	\end{itemize}
	
	\vspace{3mm}
	
 \item {\large {\bf Jefferson Laboratory Tritium Experiments and Ar(e,e'p) Experiment 2016-Present} }

 \begin{itemize}\itemsep5pt \parskip0pt \parsep0pt
  \item \textbf{Physics Analyses:}
    \begin{itemize}\itemsep5pt \parskip0pt \parsep0pt
     \item $\bullet$ Performed optics calibration procedure for path reconstruction of particle tracks through the high resolution spectrometers in JLab Hall A.
     \item $\bullet$ Performed particle identification distinguishing electrons from pions by placing physics cuts on the ADC spectra of gas Cherenkovs and electromagnetic calorimeters.
     \end{itemize}

  \item \textbf{Hardware/Software:}
    \begin{itemize}\itemsep5pt \parskip0pt \parsep0pt
     \item $\bullet$ Performed detector maintenance in JLab's Hall A emphasizing the vertical drift chambers (VDCs). 
     \item $\bullet$ Maintained and configured the Hall A data acquisition system for the high resolution spectrometers.
     \item $\bullet$ Maintained and built graphical user interfaces (GUIs) used by shift workers to operate the Hall A DAQ system as well as monitor data quality.
     \item $\bullet$ Wrote code to monitor electronics induced dead time in real-time for the Tritium experiments.
    \end{itemize}
    
   \item \textbf{Simulation:}
     \begin{itemize}\itemsep5pt \parskip0pt \parsep0pt
      \item $\bullet$ Built GEANT4 simulation of high resolution spectrometer pion rejector (calorimeter).
      \item $\bullet$ Built a Monte Carlo simulation of the Hall A VDCs to identify and reduce crosstalk between wires. 
     \end{itemize} 
     
     \item \textbf{Shift Work:}
     \begin{itemize}\itemsep5pt \parskip0pt \parsep0pt
      \item $\bullet$ Shift worker for DVCS, GMp, Ar(e,e'p), and Tritium suite of experiments. Responsibilities included: 
      	\begin{itemize}\itemsep2pt
      		\item - Monitoring data quality.
      		\item - Replaying and analyzing data. 
      		\item - Communicating with accelerator control.
      	\end{itemize} 
      \item $\bullet$ Shift leader for the Tritium experiments. Responsibilities included:
		\begin{itemize}\itemsep2pt
			\item - Coordinating workers on shift as well as with accelerator operations.
			\item - Executing the run plan laid out by the spokespeople and run coordinator. 
			\item - Troubleshooting experimental issues during runs such as equipment failures.
		\end{itemize}      
      \item $\bullet$ Cryotarget operator for Tritium suite of experiments. Responsibilities included:
      	\begin{itemize}\itemsep2pt
        		\item - Ensuring that the cryogenic $^{3}$H target maintained a proper temperature.
        		\item - Coordinating movement between different experimental targets with accelerator control.
      	\end{itemize}
     \end{itemize} 
     
     \item \textbf{Communication/Teamwork:}
		\begin{itemize}\itemsep5pt
			\item $\bullet$ Regularly gave presentations to weekly collaboration meetings.
			\item $\bullet$ Organized and led teams of graduate students in completing experimental tasks.
		\end{itemize}
     
 \end{itemize}
 
 \vspace{3mm}
 
\item {\large {\bf Jefferson Laboratory GRINCH DAQ and VETROC Board 2015-2016} }

 \begin{itemize}\itemsep5pt \parskip0pt \parsep0pt
  \item \textbf{Physics Analyses:}
    \begin{itemize}\itemsep5pt \parskip0pt \parsep0pt
     \item $\bullet$ Performed benchmark tests on JLab's new VETROC board for use as a self-triggering high-rate TDC. Parameters determined include sensitivity, data rate vs. dead time, and two-hit resolution.
     \end{itemize}

  \item \textbf{Hardware/Software:}
    \begin{itemize}\itemsep5pt \parskip0pt \parsep0pt
     \item $\bullet$ Built data acquisition (DAQ) system for GRINCH detector using the VETROC board which is capable of forming triggers in real-time for high data rates. 
     \item $\bullet$ Built a FASTBUS DAQ system for the GRINCH detector which was tested with a prototype detector to compare FASTBUS performance with VETROC performance.
     \item $\bullet$ Developed cluster finding algorithm written in VHDL to form the main GRINCH trigger using FPGAs built into the VETROC board.
     \item $\bullet$ Developed event display for GRINCH detector using ROOT. 
    \end{itemize}
    
   \item \textbf{Simulation:}
     \begin{itemize}\itemsep5pt \parskip0pt \parsep0pt
      \item $\bullet$ Built GEANT4 simulation of GRINCH prototype detector.
     \end{itemize} 
     
   \item \textbf{Communication:}
	\begin{itemize}\itemsep5pt
		\item $\bullet$ Presented results of VETROC tests to JLab collaborators.
	\end{itemize}
 \end{itemize}

\vspace{3mm}

\item {\large {\bf Jefferson Laboratory GRINCH PMT Studies 2013-2016} }

 \begin{itemize}\itemsep5pt \parskip0pt \parsep0pt
  \item \textbf{Physics Analyses:}
   
    \begin{itemize}\itemsep5pt \parskip0pt \parsep0pt
     \item $\bullet$ Analyzed a wavelength shifting (WLS) paint's effect on 9125B photomultiplier tubes to determine if the WLS paint could increase the number of UV photoelectrons detected in the GRINCH. 
     \item $\bullet$ Determined the optimal WLS paint thickness to maximize photoelectron detection using a coordinate measuring machine.
     \item $\bullet$ Collaborated with a scientist at Fermilab to determine the emission spectrum of the WLS paint to be applied to the GRINCH PMTs.
     \item $\bullet$ Calculated the gain in photoelectrons detected by PMTs due to the application of the WLS paint. 
     \end{itemize}

  \item \textbf{Hardware:}
    \begin{itemize}\itemsep5pt \parskip0pt \parsep0pt
     \item $\bullet$ Designed and built a detector to measure Cherenkov light created in a silica crystal by cosmic rays. 
     \item $\bullet$ Measured the amount of light absorbed by a WLS paint at various wavelengths using a spectrometer. 
    \end{itemize}
    
  \item \textbf{Communication:}
	\begin{itemize}\itemsep5pt
		\item $\bullet$ Collaborated with Eljen Technologies to create a custom WLS paint meeting our specific needs.
	\end{itemize}
    
 \end{itemize}

\vspace{3mm}

\item {\large {\bf College of William and Mary GRINCH Detector 2013-2016} }
 \begin{itemize}\itemsep5pt \parskip0pt \parsep0pt
  \item \textbf{Physics Analyses:}
   
    \begin{itemize}\itemsep5pt \parskip0pt \parsep0pt
     \item $\bullet$ Studied the optical properties of the mirror system for the Gas Ring-ImagiNg CHerenkov (GRINCH) detector built for experiments at the Thomas Jefferson National Accelerator Facility. 
     \item $\bullet$ Optimized photon collection by finding the ideal radius of curvature and angle for each mirror.
     \end{itemize}

  \item \textbf{Hardware:}
    \begin{itemize}\itemsep5pt \parskip0pt \parsep0pt
     \item $\bullet$ Built the mirror system to be used in the GRINCH Cherenkov detector.
     \item $\bullet$ Designed and implemented procedure to align the mirror system to direct light onto the PMT array using a laser mounted on a 360$^\circ$ linear rail system.
    \end{itemize}
    
   \item \textbf{Simulation:}
     \begin{itemize}\itemsep5pt \parskip0pt \parsep0pt
      \item $\bullet$ Created a simulation of the GRINCH detector mirror system in Mathematica to investigate anomalous experimental results.
     \end{itemize} 
 
\end{itemize}

\vspace{3mm}

\item {\large {\bf College of William and Mary SEOP 2012-2013} }
 \begin{itemize}\itemsep5pt \parskip0pt \parsep0pt
  \item \textbf{Physics Analyses:}
   
    \begin{itemize}\itemsep5pt
     \item $\bullet$ Developed code in LabVIEW to make a SEOP apparatus perform a `spin flip' of polarized $^{3}$He.
     \item $\bullet$ Studied $^{3}$He polarized by SEOP as it underwent `spin flips' using NMR.
     \end{itemize}

  \item \textbf{Hardware:}
    \begin{itemize}\itemsep5pt
     \item $\bullet$ Built laser interlock system for SEOP pumping lasers to automatically regulate cell temperatures. 
    \end{itemize}
 
\end{itemize}

\vspace{3mm}

\item {\large {\bf Drake University 2011-2012} }
 \begin{itemize}\itemsep5pt \parskip0pt \parsep0pt
  \item \textbf{Physics Analyses:} 
    \begin{itemize}\itemsep5pt \parskip0pt \parsep0pt
     \item $\bullet$ Studied the mass renormalization of the electron due to self-coupling interactions numerically by means of a staggered-leap-frog algorithm.   
     \end{itemize}
     
    \item \textbf{Communication:}
	\begin{itemize}\itemsep5pt
		\item $\bullet$ Presented results to the annual meeting of the Iowa Academy of Sciences in April 2012.
	\end{itemize}
     
\end{itemize}

\vspace{3mm}

\item {\large {\bf University of Southern Illinois Carbondale SEOP 2011 } }
 \begin{itemize}\itemsep5pt \parskip0pt \parsep0pt
  \item \textbf{Physics Analyses:} 
    \begin{itemize}\itemsep5pt \parskip0pt \parsep0pt
     \item $\bullet$ Performed Spin Exchange Optical Pumping (SEOP) on $^{129}$Xe to hyperpolarize the $^{129}$Xe gas.
     \item $\bullet$ Optimized $^{129}$Xe polarization via pumping laser calibrations studied via NMR spectroscopy. 
     \item $\bullet$ Developed LabVIEW GUI and control software for pumping laser. 
     \end{itemize}

  \item \textbf{Hardware:}
    \begin{itemize}\itemsep5pt \parskip0pt \parsep0pt
     \item $\bullet$ Built SEOP apparatus used to polarize $^{129}$Xe used for medical imaging research. 
     \item $\bullet$ The apparatus successfully polarized $^{129}$Xe via rubidium SEOP replacing an expensive proprietary polarization apparatus made by GE. 
    \end{itemize}
\end{itemize}

\end{itemize} 

{\Large{Awards and Honors}}
\begin{itemize} \itemsep5pt \parskip0pt \parsep0pt
	\item $\bullet$ \textbf{Visiting Fellow at the National Security Institute at George Mason University's Antonin Scalia Law School:} Working to craft policy recommendations from a technologist's perspective, 2019.
	\item $\bullet$ \textbf{Roy L. Champion Physics Award:} College of William and Mary for research excellence, 2019.
	\item $\bullet$ \textbf{National Security Institute Technologist Fellowship:} George Mason University's Antonin Scalia Law School, 2019.
	\item $\bullet$ \textbf{Jefferson Lab Run-A-Round:} First place in the men's 26-35 division run, May 2018 and May 2019.
	\item $\bullet$ \textbf{Jefferson Science Associates Graduate Fellowship:} Awarded to study electron scattering off of $^3$H and $^3$He at the Thomas Jefferson National Accelerator Facility, 2017-2018.
	\item $\bullet$ \textbf{Deans List:} Drake University 2008-2012.
	\item $\bullet$ \textbf{Best Student Organized Programming:} Runner up, Drake University 2008.
	\item $\bullet$ \textbf{Physics Scholarship:} Drake University 2008-2012
	\item $\bullet$ \textbf{Presidential Scholarship:} Drake University 2008-2012.
  
\end{itemize}

%\section*{Teaching Experience}
{\Large{Teaching Experience}}
\begin{itemize} \itemsep5pt \parskip0pt \parsep0pt
  \item $\bullet$ \textbf{Teaching Assistant:} Astronomy Laboratory, Fall Semester 2012.
  \item $\bullet$ \textbf{Physics Tutor:} Tutor of Undergraduate Physics, Fall Semester 2010 - Spring Semester 2011.
\end{itemize}

%\section*{Professional Development}
{\Large{Professional Development}}
\begin{itemize}\itemsep5pt \parskip0pt \parsep0pt
 \item $\bullet$ \textbf{National Security Institute at GMU Technologist Fellowship:} Fellowship and training awarded to teach technologists how to best interact with government to advocate for science and technology policies, 2018-2019.
 \item $\bullet$ \textbf{Hampton University Graduate Summer Program:} Summer school hosted by Jefferson Lab for Graduate students in experimental and theoretical nuclear physics, Summer 2016.
 \item $\bullet$ \textbf{REU Researcher:} Studied spin exchange optical pumping of $^{129}$Xe for medical imaging at the University of Southern Illinois Carbondale, Summer 2011.
\end{itemize}

%\section*{Professional Service}
{\Large{Professional Service/Volunteer Work}}
\begin{itemize}\itemsep5pt \parskip0pt \parsep0pt 
  \item $\bullet$ \textbf{W$\&$M Diversity Advisory Committee Co-Chair:} Co-chaired the newly formed DAC tasked with improving the department's climate and increasing diversity. College of William and Mary, August 2018-2019.
  	\begin{itemize}
  		\item - Created and implemented a Diversity Plan aimed at recruiting women and underrepresented minorities.
  	\end{itemize}
  \item $\bullet$ \textbf{W$\&$M Climate Steering Committee:} Formed to assess social climate issues in the physics department and propose policy to resolve issues. College of William and Mary, 2015-2018.
  	\begin{itemize}\itemsep2pt
  		\item - Founding member of CSC, 2015-2016. 
  		\item - Co-chair of CSC, November 2016 - August 2018. 
  			\begin{itemize}
  			\item i. Wrote a Code of Conduct and Statement of Values which the W$\&$M physics department then ratified.
  			\item ii. Created regular social climate surveys to flag climate issues and track the efficacy of new policies. 
  		 	\end{itemize}
  	\end{itemize}
  	  \item $\bullet$ \textbf{Treasurer:} Society of Physics Students, Drake University, 2011-2012. 
\end{itemize}

%\section*{Professional Development}
{\Large{Professional Organizations}}
\begin{itemize}\itemsep5pt \parskip0pt \parsep0pt
 \item $\bullet$ \textbf{American Geophysical Union}
 \item $\bullet$ \textbf{American Physical Society}
\end{itemize}

%\section*{Publications}
{\Large{Publications}}
\subsection*{Papers}
\begin{enumerate}\itemsep1pt \parskip0pt \parsep0pt
\item Nikolaou, \textit{et al.} \textit{``Near-unity nuclear polarization with an open-source 129Xe hyperpolarizer for NMR and MRI.''}
PNAS 2013, August 14, 2013, \url{http://dx.doi.org/10.1073/pnas.1306586110}.

\item Dai \textit{et al.} \textit{``First Measurement of the Ti$(e,e^{'})$X Cross Section at Jefferson Lab.''} Physical Review C, vol. 98, no. 1, 2018, \url{http://dx.doi.org/10.1103/physrevc.98.014617}.

%\item Dai, \textit{et al.} \textit{``First Measurement of the Ar$(e,e^{'})$X Cross Section at Jefferson Lab.''} October 24, 2018, \\ \url{https://arxiv.org/abs/1810.10575}.

\item Dai, \textit{et al.} \textit{``First Measurement of the Ar$(e,e^{'})$X Cross Section at Jefferson Lab.''} October 29, 2018, Phys. Rev. C 99, 054608, \url{https://doi.org/10.1103/PhysRevC.99.054608}.

%\item S. N. Santiesteban, \textit{et al.} \textit{``Density Changes in Low Pressure Gas Targets for Electron Scattering Experiments.''} November 26, 2018, \url{https://arxiv.org/abs/1811.12167}.

\item S. N. Santiesteban, \textit{et al.} \textit{``Density Changes in Low Pressure Gas Targets for Electron Scattering Experiments.''} October 1, 2019, Nuclear Inst. and Methods in Physics Research A, vol. 940, 351-358, \url{https://doi.org/10.1016/j.nima.2019.06.025}.

%\item R. Cruz-Torres, \textit{et al.} \textit{``Probing nucleon momentum distributions in A = 3 nuclei via $^3$He and $^3$H(e, e'p) measurements.''} December 18, 2018, Phys. Rev. Lett. Preprint.

\item R. Cruz-Torres, \textit{et al.} \textit{``Comparing proton momentum distributions in A=2 and 3 nuclei via $^2$H, $^3$H, and $^3$He(e,e'p) measurements''} October 10, 2019, Phys. Lett. B, vol. 797, \url{https://doi.org/10.1016/j.physletb.2019.134890}.

\item Murphy, \textit{et al.} \textit{``Measurement of the Cross Sections for Inclusive Electron Scattering in the E12-14-012 Experiment at Jefferson Lab''} November 11, 2019, Phys. Rev. C 100, 054606, \url{https://doi.org/10.1103/PhysRevC.100.054606}.

\item R. Cruz-Torres, \textit{et al.} \textit{``Probing Few-Body Nuclear Dynamics via $^{3}\mathrm{H}$ and $^{3}\mathrm{He}$ $(e,{e}^{\ensuremath{'}}p)\mathrm{pn}$ Cross-Section Measurements''} May 26, 2020, Phys. Rev. Lett., vol. 124, 212501, \url{https://doi.org/10.1103/PhysRevLett.124.212501}.

\item S. Barcus, \textit{et al.} \textit{``How Analytic Choices Can Affect the Extraction of Electromagnetic Form Factors from Elastic Electron Scattering Cross Section Data''} July 10, 2020. Phys. Rev. C 102, 015205, \url{https://journals.aps.org/prc/abstract/10.1103/PhysRevC.102.015205}.

\item M.E. Christy, \textit{et al.} \textit{``Two-Photon Exchange in Electron-Proton Elastic Scattering at Large Four-Momentum Transfer''} not sure which journal yet, \url{not online yet}.%Second draft GMp12 6/13/2020 email from Bogdan

\item M. Dlamini, \textit{et al.} \textit{``Deep Exclusive Electroproduction of $\pi^0$ at High $Q^2$ in Valence Regime''} ArXiv not sure which journal eventually, \url{https://arxiv.org/abs/2011.11125}.%First draft DVCS12 8/8/2020 email from Julie Roche

\item D. Adhikari, \textit{et al.} \textit{``An Accurate Determination of the Neutron Skin Thickness of $^208$Pb Through Parity-Violation in Electron Scattering''} not sure which journal yet, \url{not online yet}.%PREX2 email from Kent Pashke 2/12/2021.

\item D. Abrams, \textit{et al.} \textit{``Measurement of the Nucleon $F_2^n/F_2^p$Structure Function Ratio by the JLab MARATHON Tritium/Helium-3 DIS Experiment''} not sure which journal yet, \url{not online yet}.%MARATHON F2n/F2p draft email from GERASSIMOS PETRATOS on 2/15/2021.
 
%\subsection*{Papers Under Review}

\subsection*{Papers In Progress}
\item Barcus, \textit{et al.} \textit{``Extraction and Parametrization of Isobaric Trinucleon Elastic Cross Sections and Form Factors''} journal TBD.

\end{enumerate}
\subsection*{Talks and Posters}


	\begin{enumerate}\itemsep1pt \parskip2pt \parsep0pt
		\setcounter{enumi}{0}
		%\begin{itemize}\itemsep1pt \parskip0pt \parsep0pt
		\item \textit{``Dynamic Mass of Self-Interacting Fermions''}. Annual Meeting of the Iowa Academy of Sciences, Mason City, IA, April 2012.

		\item \textit{"9125B ET Photomultiplier Tubes with a Wavelength Shifting Paint for a Gas Cherenkov Counter"} Talk presented to the 2015 April Meeting of the APS. Balitmore, MD, April 11 2015. 
		
		\item \textit{``New $^3$He Form Factor Measurements and Global Fits''}. Photonuclear Reactions Gordon Research Conference, Holderness, NH, August 2018.
		
		\item \textit{``New $^3$He Elastic Cross Section Measurements and Global Fits''}. Talk presented to the American Physical Society Division of Nuclear Physics Meeting, Waikoloa Village, HI, October 27 2018.
		
		\item \textit{``Extraction and Parametrization of Isobaric Trinucleon Elastic Cross Sections and Form Factors''}. Postdoctoral Seminar presented to Jefferson Lab, Newport News, VA, February 5 2019.
		
		\item \textit{``Extraction and Parametrization of Isobaric Trinucleon Elastic Cross Sections and Form Factors''}. Dissertation presented to the College of William $\&$ Mary and the public, Williamsburg, VA, March 18 2019.
		
		\item \textit{``Hadron Calorimeter Update''}. Talk presented to Super BigBite Collaboration, Newport News, VA, May 9 2019.
		
		\item \textit{``High Q$^2$ Elastic Scattering on Three-body Nuclei''}. Seminar presented to Jefferson Lab, Newport News, VA, May 15 2019.
		
		\item \textit{``First High $Q^2$ $^3$He Form Factor Measurements using Polarization Observables''}. Proposal presented to $A_1^n$/$d_2^n$ Collaboration Meeting, Newport News, VA, July 24 2019.
		
		\item \textit{``Unravelling the $^3$He Electromagnetic Form Factors''}. Talk presented to the APS Division of Nuclear Physics Meeting, Arlington, VA, October 16 2019.
		
		\item \textit{``Design and Commissioning Results of the New HCAL-J Hadron Calorimeter for Upcoming Nucleon Form Factor Experiments at JLab''}. Talk presented to the APS Division of Nuclear Physics Meeting, Virtual, April 20 2020.
	
	\end{enumerate}


\bigskip
% Footer
\begin{center}
  \begin{footnotesize}
    Last updated: \today \\
    \href{\footerlink}{\texttt{\footerlink}}
  \end{footnotesize}
\end{center}
}
\end{document}