% LaTeX Curriculum Vitae Template
%
% Copyright (C) 2004-2009 Jason Blevins <jrblevin@sdf.lonestar.org>
% http://jblevins.org/projects/cv-template/
%
% You may use use this document as a template to create your own CV
% and you may redistribute the source code freely. No attribution is
% required in any resulting documents. I do ask that you please leave
% this notice and the above URL in the source code if you choose to
% redistribute this file.
%
% If you have never used LaTeX before, talk you your advisor about
% how to compile this into a PDF.

\documentclass[letterpaper,10pt]{article}

\usepackage{hyperref}
\usepackage[hmargin=1.5cm,vmargin=1.5cm]{geometry}
% Comment the following lines to use the default Computer Modern font
% instead of the Palatino font provided by the mathpazo package.
% Remove the 'osf' bit if you don't like the old style figures.
\usepackage[T1]{fontenc}
\usepackage[sc,osf]{mathpazo}

\usepackage{setspace}

\usepackage{mdwlist}

% Set your name here
\def\name{Scott K. Barcus}

% Replace this with a link to your CV if you like, or set it empty
% (as in \def\footerlink{}) to remove the link in the footer:
\def\footerlink{}

% The following metadata will show up in the PDF properties
\hypersetup{
  colorlinks = true,
  urlcolor = black,
  pdfauthor = {\name},
  pdfkeywords = {CV, physics},
  pdftitle = {\name: Curriculum Vitae},
  pdfsubject = {Curriculum Vitae},
  pdfpagemode = UseNone
}

\geometry{
  body={6.5in, 8.5in},
  left=1.0in,
  top=1.0in
}

% Customize page headers
\pagestyle{myheadings}
\markright{\name}
\thispagestyle{empty}

\usepackage{titlesec}

% Custom section fonts
\usepackage{sectsty}
\sectionfont{\rmfamily\mdseries\Large}
\subsectionfont{\rmfamily\mdseries\large}
%\subsectionfont{\rmfamily\mdseries\itshape\large}

% Other possible font commands include:
% \ttfamily for teletype,
% \sffamily for sans serif,
% \bfseries for bold,
% \scshape for small caps,
% \normalsize, \large, \Large, \LARGE sizes.

% Don't indent paragraphs.
\setlength\parindent{0em}

% Make lists without bullets
\renewenvironment{itemize}{
  \begin{list}{}{
    \setlength{\leftmargin}{1.5em}
  }
}{
  \end{list}
}

\begin{document}
\renewcommand{\baselinestretch}{0.0}{
% Place name at left
{\huge \name}

% Alternatively, print name centered and bold:
%\centerline{\huge \bf \name}

\vspace{0.25in}

\begin{minipage}{0.45\linewidth}
  \href{http://www.wm.edu/}{101 Lake Powell Road Apartment L} \\
  Williamsburg, VA 23185 \\
\end{minipage}
\begin{minipage}{0.45\linewidth}
  \begin{tabular}{ll}
    Phone: & (651) 324-9205 \\
    Email: & \href{mailto:me@email.wm.edu}{\tt skbarcus@email.wm.edu} \\
  \end{tabular}
\end{minipage}

%\titlespacing{\section}{0pt}{\parskip}{-\parskip}
%\titlespacing\section*{0pt}{12pt plus 4pt minus 2pt}{0pt plus 2pt minus 2pt}
%{0pt}{5.5ex plus 1ex minus .2ex}{4.3ex plus .2ex}

%\setlength{\parskip}{\baselineskip}
%\newlength{\aftersection}
%\setlength{\aftersection}{0pt}
%\addtolength{\aftersection}{0pt}%-1\parskip}
%\newlength{\beforesection}
%\setlength{\beforesection}{0pt}
%\addtolength{\beforesection}{0pt}%-1\parskip}
%\titlespacing*{\section}{-50pt}{\aftersection}{\beforesection}

%\section*{Education}
{\Large{Education}}
\begin{itemize}\itemsep5pt \parskip0pt \parsep0pt
 %\singlespacing{
  \item B.S. Physics, B.S. Mathematics, B.S. Astronomy, Drake University, 2012.
  \item M.S. Physics, College of William and Mary, 2014.
  \item Ph.D. Physics, College of William and Mary, expected January 2019.
\end{itemize}

%\clearpage

%\section*{Research Experience}
{\Large{Research Experience}}
\begin{itemize}\itemsep5pt \parskip0pt \parsep0pt

\item {\large {\bf University of Southern Illinois Carbondale SEOP 2011 } }
 \begin{itemize}\itemsep5pt \parskip0pt \parsep0pt
  \item \textbf{Physics Analyses:} 
    \begin{itemize}\itemsep5pt \parskip0pt \parsep0pt
     \item $\bullet$ Performed Spin Exchange Optical Pumping (SEOP) on $^{129}$Xe to hyperpolarize the $^{129}$Xe gas.
     \item $\bullet$ Optimized $^{129}$Xe polarization via pumping laser calibrations studied via NMR spectroscopy. 
     \item $\bullet$ Developed LabVIEW GUI and control software for pumping laser. 
     \end{itemize}

  \item \textbf{Hardware:}
    \begin{itemize}\itemsep5pt \parskip0pt \parsep0pt
     \item $\bullet$ Built SEOP apparatus used to polarize $^{129}$Xe used for medical imaging research. 
     \item $\bullet$ The apparatus successfully polarized $^{129}$Xe via rubidium SEOP replacing an expensive proprietary polarization apparatus made by GE. 
    \end{itemize}
 
\end{itemize}

\item {\large {\bf Drake University 2011-2012} }
 \begin{itemize}\itemsep5pt \parskip0pt \parsep0pt
  \item \textbf{Physics Analyses:} 
    \begin{itemize}\itemsep5pt \parskip0pt \parsep0pt
     \item $\bullet$ Studied the mass renormalization of the electron due to self-coupling interactions numerically by means of a staggered-leap-frog algorithm.   
     \end{itemize}

\end{itemize}

\item {\large {\bf College of William and Mary SEOP 2012-2013} }
 \begin{itemize}\itemsep5pt \parskip0pt \parsep0pt
  \item \textbf{Physics Analyses:}
   
    \begin{itemize}\itemsep5pt
     \item $\bullet$ Developed code in LabVIEW to make a SEOP apparatus perform a `spin flip' of polarized $^{3}$He.
     \item $\bullet$ Studied $^{3}$He polarized by SEOP as it underwent `spin flips' using NMR.
     \end{itemize}

  \item \textbf{Hardware:}
    \begin{itemize}\itemsep5pt
     \item $\bullet$ Built laser interlock system for SEOP pumping lasers to automatically regulate cell temperatures. 
    \end{itemize}
 
\end{itemize}

\item {\large {\bf College of William and Mary GRINCH Detector 2013-2016} }
 \begin{itemize}\itemsep5pt \parskip0pt \parsep0pt
  \item \textbf{Physics Analyses:}
   
    \begin{itemize}\itemsep5pt \parskip0pt \parsep0pt
     \item $\bullet$ Studied the optical properties of the mirror system for the Gas Ring-ImagiNg CHerenkov (GRINCH) detector built for experiments at the Thomas Jefferson National Accelerator Facility. 
     \item $\bullet$ Optimized photon collection by finding the ideal radius of curvature and angle for each mirror.
     \end{itemize}

  \item \textbf{Hardware:}
    \begin{itemize}\itemsep5pt \parskip0pt \parsep0pt
     \item $\bullet$ Built the mirror system to be used in the GRINCH Cherenkov detector.
     \item $\bullet$ Designed and implemented procedure to align the mirror system to direct light onto the PMT array using a laser mounted on a 360$^\circ$ linear rail system.
    \end{itemize}
    
   \item \textbf{Simulation:}
     \begin{itemize}\itemsep5pt \parskip0pt \parsep0pt
      \item $\bullet$ Created a simulation of the GRINCH detector mirror system in Mathematica to investigate anomalous experimental results.
     \end{itemize} 
 
\end{itemize}

\item {\large {\bf Jefferson National Laboratory GRINCH PMT Studies 2013-2016} }

 \begin{itemize}\itemsep5pt \parskip0pt \parsep0pt
  \item \textbf{Physics Analyses:}
   
    \begin{itemize}\itemsep5pt \parskip0pt \parsep0pt
     \item $\bullet$ Analyzed a wavelength shifting (WLS) paint's effect on 9125B photomultiplier tubes to determine if the WLS paint could increase the number of UV photoelectrons detected in the GRINCH. 
     \item $\bullet$ Determined the optimal WLS paint thickness to maximize photoelectron detection using a coordinate measuring machine.
     \item $\bullet$ Collaborated with a scientist at Fermilab to determine the emission spectrum of the WLS paint to be applied to the GRINCH PMTs.
     \item $\bullet$ Calculated the gain in photoelectrons detected by PMTs due to the application of the WLS paint. 
     \end{itemize}

  \item \textbf{Hardware:}
    \begin{itemize}\itemsep5pt \parskip0pt \parsep0pt
     \item $\bullet$ Designed and built a detector to measure Cherenkov light created in a silica crystal by cosmic rays. 
     \item $\bullet$ Measured the amount of light absorbed by a WLS paint at various wavelengths using a spectrometer. 
    \end{itemize}
 \end{itemize}
 
\item {\large {\bf Jefferson National Laboratory GRINCH DAQ and VETROC Board 2015-2016} }

 \begin{itemize}\itemsep5pt \parskip0pt \parsep0pt
  \item \textbf{Physics Analyses:}
    \begin{itemize}\itemsep5pt \parskip0pt \parsep0pt
     \item $\bullet$ Performed benchmark tests on JLab's new VETROC board for use as a self-triggering high-rate TDC. Parameters determined include sensitivity, data rate vs. dead time, and two-hit resolution.
     \end{itemize}

  \item \textbf{Hardware/Software:}
    \begin{itemize}\itemsep5pt \parskip0pt \parsep0pt
     \item $\bullet$ Built data acquisition (DAQ) system for GRINCH detector using the VETROC board which is capable of forming triggers in real-time for high data rates. 
     \item $\bullet$ Built a FASTBUS DAQ system for the GRINCH detector which was tested with a prototype detector to compare FASTBUS performance with VETROC performance.
     \item $\bullet$ Developed cluster finding algorithm written in VHDL to form the main GRINCH trigger using FPGAs built into the VETROC board.
     \item $\bullet$ Developed event display for GRINCH detector using ROOT. 
    \end{itemize}
    
   \item \textbf{Simulation:}
     \begin{itemize}\itemsep5pt \parskip0pt \parsep0pt
      \item $\bullet$ Built GEANT4 simulation of GRINCH prototype detector.
     \end{itemize} 
 \end{itemize}
 
 \item {\large {\bf Jefferson National Laboratory Tritium Experiments and Ar(e,e'p) Experiment 2016-Present} }

 \begin{itemize}\itemsep5pt \parskip0pt \parsep0pt
  \item \textbf{Physics Analyses:}
    \begin{itemize}\itemsep5pt \parskip0pt \parsep0pt
     \item $\bullet$ Performed optics calibration procedure for path reconstruction of particle tracks through the high resolution spectrometers in JLab Hall A.
     \item $\bullet$ Performed particle identification distinguishing electrons from pions by placing physics cuts on the ADC spectra of gas Cherenkovs and electromagnetic calorimeters.
     \end{itemize}

  \item \textbf{Hardware/Software:}
    \begin{itemize}\itemsep5pt \parskip0pt \parsep0pt
     \item $\bullet$ Performed detector maintenance in JLab's Hall A emphasizing the vertical drift chambers (VDCs). 
     \item $\bullet$ Maintained and configured the Hall A data acquisition system for the high resolution spectrometers.
     \item $\bullet$ Maintained and built graphical user interfaces (GUIs) used by shift workers to operate the Hall A DAQ system as well as monitor data quality.
     \item $\bullet$ Wrote code to monitor electronics induced dead time in real-time for the Tritium experiments.
    \end{itemize}
    
   \item \textbf{Simulation:}
     \begin{itemize}\itemsep5pt \parskip0pt \parsep0pt
      \item $\bullet$ Built GEANT4 simulation of high resolution spectrometer pion rejector (calorimeter).
      \item $\bullet$ Built a Monte Carlo simulation of the Hall A VDCs to identify and reduce crosstalk between wires. 
     \end{itemize} 
     
     \item \textbf{Shift Work:}
     \begin{itemize}\itemsep5pt \parskip0pt \parsep0pt
      \item $\bullet$ Shift worker for DVCS, GMp, Ar(e,e'p), and Tritium suite of experiments. Responsibilities included: 
      	\begin{itemize}\itemsep2pt
      		\item - Monitoring data quality.
      		\item - Replaying and analyzing data. 
      		\item - Communicating with accelerator control.
      	\end{itemize} 
      \item $\bullet$ Shift leader for the Tritium experiments. Responsibilities included:
		\begin{itemize}\itemsep2pt
			\item - Coordinating workers on shift as well as with accelerator operations.
			\item - Executing the run plan laid out by the spokespeople and run coordinator. 
			\item - Troubleshooting experimental issues during runs such as equipment failures.
		\end{itemize}      
      \item $\bullet$ Cryotarget operator for Tritium suite of experiments. Responsibilities included:
      	\begin{itemize}\itemsep2pt
        		\item - Ensuring that the cryogenic $^{3}$H target maintained a proper temperature.
        		\item - Coordinating movement between different experimental targets with accelerator control.
      	\end{itemize}
     \end{itemize} 
     
 \end{itemize}

\item {\large {\bf Jefferson National Laboratory Elastic $^{3}$He Cross Section Extraction 2017-Present} }
	\begin{itemize}\itemsep5pt \parskip0pt \parsep0pt
		\item \textbf{Cross Section Extraction:}
			\begin{itemize}\itemsep5pt
				\item $\bullet$ Calculated yield of electrons elastically scattered off gaseous $^{3}$He target during experiment E08-014.   
				\item $\bullet$ Performed a background subtraction of the target cell's aluminium walls leaving only events scattered from $^{3}$He.
				\item $\bullet$ Determined the efficiencies of detectors including the VDCs, gas Cherenkov, and pion rejectors.
				\item $\bullet$ Calculated the cross section for elastic scattering off of $^{3}$He in the high Q$^2$ region of Q$^2 \approx 34.2$ fm$^{-2}$.
			\end{itemize}
		\item \textbf{Simulation:}
			\begin{itemize}\itemsep5pt
				\item $\bullet$ Performed Monte Carlo simulations of elastic electron scattering off $^{3}$He with JLab's SIMC package. 
				\item $\bullet$ Used Monte Carlo simulation results to determine the acceptance for elastically scattered electrons of the Hall A high resolution spectrometers during experiment E08-014. 
				\item $\bullet$ Modelled the radiative effects present in experimental data using SIMC to apply radiative corrections to the $^{3}$He cross section.
			\end{itemize}
		\item \textbf{Global Fits and Form Factors:}
			\begin{itemize}\itemsep5pt
				\item $\bullet$ Compiled a database containing the world data on $^{3}$H, $^{3}$He, and $^{4}$He cross sections and form factors.
				\item $\bullet$ Developed code to fit $^{3}$H and $^{3}$He cross section world data with a sum of Gaussians (SOG) technique. 
				\item $\bullet$ Extracted charge and magnetic form factors from these global fits incorporating modern data. 
				\item $\bullet$ Calculated charge densities for $^{3}$H and $^{3}$He from new fits and extracted charge radii. 
			\end{itemize}
	\end{itemize}

\end{itemize} 

{\Large{Awards and Honors}}
\begin{itemize} \itemsep5pt \parskip0pt \parsep0pt
  \item $\bullet$ \textbf{Jefferson Science Associates Graduate Fellowship:} Awarded to study electron scattering off of $^3$H and $^3$He at the Thomas Jefferson National Accelerator Facility, 2017-2018.
  \item $\bullet$ \textbf{Jefferson Lab Run-A-Round:} First place in the men's 26-35 division fun run, May 2018. 
\end{itemize}

%\section*{Teaching Experience}
{\Large{Teaching Experience}}
\begin{itemize} \itemsep5pt \parskip0pt \parsep0pt
  \item $\bullet$ \textbf{Physics Tutor:} Tutor of Undergraduate Physics, Fall Semester 2010 - Spring Semester 2011.
  \item $\bullet$ \textbf{Teaching Assistant:} Astronomy Laboratory, Fall Semester 2012.
\end{itemize}

%\section*{Professional Development}
{\Large{Professional Development}}
\begin{itemize}\itemsep5pt \parskip0pt \parsep0pt
 \item $\bullet$ \textbf{REU Researcher:} Studied spin exchange optical pumping of $^{129}$Xe for medical imaging at the University of Southern Illinois Carbondale, Summer 2011.
 \item $\bullet$ \textbf{Hampton University Graduate Summer Program:} Summer school hosted by Jefferson Lab for Graduate students in experimental and theoretical nuclear physics, Summer 2016.
\end{itemize}

%\section*{Professional Service}
{\Large{Professional Service}}
\begin{itemize}\itemsep5pt \parskip0pt \parsep0pt
  \item $\bullet$ \textbf{Treasurer:} Society of Physics Students, Drake University, 2011-2012.
  \item $\bullet$ \textbf{W$\&$M Climate Steering Committee Member:} Formed to assess social climate issues in the physics department and propose policy to resolve issues, College of William and Mary, 2015-2016.
  \item $\bullet$ \textbf{W$\&$M Climate Steering Committee Co-Chair:} Co-chair of the committee, College of William and Mary, November 2016-present.
  	\begin{itemize}\itemsep2pt
  		\item - Wrote a Code of Conduct and Statement of Values which the W$\&$M physics department then ratified.
  		\item - Created regular social climate surveys to flag climate issues and track the efficacy of new policies. 
  	\end{itemize}
  \item $\bullet$ \textbf{W$\&$M Diversity Advisory Committee Co-Chair:} Co-chaired the newly formed DAC tasked with improving the department's climate and increasing diversity, College of William and Mary, August 2018-present.
  	\begin{itemize}
  		\item - Created and implemented a Diversity Plan aimed at recruiting women and underrepresented minorities.
  	\end{itemize}
\end{itemize}

%\section*{Publications}
{\Large{Publications}}
\subsection*{Papers}
\begin{enumerate}\itemsep1pt \parskip0pt \parsep0pt
\item Nikolaou, \textit{et al.} \textit{``Near-unity nuclear polarization with an open-source 129Xe hyperpolarizer for NMR and MRI.''}
PNAS 2013, August 14, 2013, doi:10.1073/pnas.1306586110.

\item Dai \textit{et al.} \textit{``First Measurement of the Ti$(e,e^{'})$X Cross Section at Jefferson Lab.''} Physical Review C, vol. 98, no. 1, 2018, doi:10.1103/physrevc.98.014617.

\item Dai, \textit{et al.} \textit{``First Measurement of the Ar$(e,e^{'})$X Cross Section at Jefferson Lab.''} October 24, 2018, \\https://arxiv.org/abs/1810.10575.

\item S. N. Santiesteban, \textit{et al.} \textit{``Density Changes in Low Pressure Gas Targets for Electron Scattering Experiments.''} November 26, 2018, https://arxiv.org/abs/1811.12167

\end{enumerate}
\subsection*{Talks and Posters}


	\begin{enumerate}\itemsep1pt \parskip2pt \parsep0pt
		\setcounter{enumi}{4}
		%\begin{itemize}\itemsep1pt \parskip0pt \parsep0pt
		\item \textit{``Dynamic Mass of Self-Interacting Fermions''}. Annual Meeting of the Iowa Academy of Sciences, Mason City, IA, April 2012.

		\item \textit{"9125B ET Photomultiplier Tubes with a Wavelength Shifting Paint for a Gas Cherenkov Counter"} Talk presented to the 2015 April Meeting of the APS. Balitmore, MD, April 11 2015. 
		
		\item \textit{``New $^3$He Form Factor Measurements and Global Fits''}. Photonuclear Reactions Gordon Research Conference, Holderness, NH, August 2018.
		
		\item \textit{``New $^3$He Elastic Cross Section Measurements and Global Fits''}. Talk presented to the American Physical Society Division of Nuclear Physics Meeting, Waikoloa Village, HI, October 27 2018.
	\end{enumerate}


\bigskip
% Footer
\begin{center}
  \begin{footnotesize}
    Last updated: \today \\
    \href{\footerlink}{\texttt{\footerlink}}
  \end{footnotesize}
\end{center}
}
\end{document}