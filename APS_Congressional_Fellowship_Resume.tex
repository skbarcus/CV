% LaTeX Curriculum Vitae Template
%
% Copyright (C) 2004-2009 Jason Blevins <jrblevin@sdf.lonestar.org>
% http://jblevins.org/projects/cv-template/
%
% You may use use this document as a template to create your own CV
% and you may redistribute the source code freely. No attribution is
% required in any resulting documents. I do ask that you please leave
% this notice and the above URL in the source code if you choose to
% redistribute this file.
%
% If you have never used LaTeX before, talk you your advisor about
% how to compile this into a PDF.

\documentclass[letterpaper,10pt]{article}

\usepackage{hyperref}
\usepackage[hmargin=1.5cm,vmargin=1.5cm]{geometry}
% Comment the following lines to use the default Computer Modern font
% instead of the Palatino font provided by the mathpazo package.
% Remove the 'osf' bit if you don't like the old style figures.
\usepackage[T1]{fontenc}
\usepackage[sc,osf]{mathpazo}

\usepackage{setspace}

\usepackage{mdwlist}

% Set your name here
\def\name{Scott K. Barcus}

% Replace this with a link to your CV if you like, or set it empty
% (as in \def\footerlink{}) to remove the link in the footer:
\def\footerlink{}

% The following metadata will show up in the PDF properties
\hypersetup{
  colorlinks = true,
  urlcolor = black,
  pdfauthor = {\name},
  pdfkeywords = {CV, physics},
  pdftitle = {\name: Curriculum Vitae},
  pdfsubject = {Curriculum Vitae},
  pdfpagemode = UseNone
}

\geometry{
  body={6.5in, 8.5in},
  left=1.0in,
  top=1.0in
}

% Customize page headers
\pagestyle{myheadings}
\markright{\name}
\thispagestyle{empty}

\usepackage{titlesec}

% Custom section fonts
\usepackage{sectsty}
\sectionfont{\rmfamily\mdseries\Large}
\subsectionfont{\rmfamily\mdseries\itshape\large}

% Other possible font commands include:
% \ttfamily for teletype,
% \sffamily for sans serif,
% \bfseries for bold,
% \scshape for small caps,
% \normalsize, \large, \Large, \LARGE sizes.

% Don't indent paragraphs.
\setlength\parindent{0em}

% Make lists without bullets
\renewenvironment{itemize}{
  \begin{list}{}{
    \setlength{\leftmargin}{1.5em}
  }
}{
  \end{list}
}

\begin{document}
\renewcommand{\baselinestretch}{0.0}{
% Place name at left
{\huge \name}

% Alternatively, print name centered and bold:
%\centerline{\huge \bf \name}

\vspace{0.25in}
\hrule height 0.8pt \relax 
\vspace{3mm}
\begin{minipage}{0.45\linewidth}
  \href{http://www.wm.edu/}{101 Lake Powell Road Apartment L} \\
  Williamsburg, VA 23185 \\
\end{minipage}
\begin{minipage}{0.45\linewidth}
  \begin{tabular}{ll}
    Phone: & 651-324-9205 \\
    Email: & \href{mailto:me@email.wm.edu}{skbarcus@email.wm.edu} \\ %\tt
  \end{tabular}
\end{minipage}

%\titlespacing{\section}{0pt}{\parskip}{-\parskip}
%\titlespacing\section*{0pt}{12pt plus 4pt minus 2pt}{0pt plus 2pt minus 2pt}
%{0pt}{5.5ex plus 1ex minus .2ex}{4.3ex plus .2ex}

%\setlength{\parskip}{\baselineskip}
%\newlength{\aftersection}
%\setlength{\aftersection}{0pt}
%\addtolength{\aftersection}{0pt}%-1\parskip}
%\newlength{\beforesection}
%\setlength{\beforesection}{0pt}
%\addtolength{\beforesection}{0pt}%-1\parskip}
%\titlespacing*{\section}{-50pt}{\aftersection}{\beforesection}
\hrule height 0.8pt \relax 
\vspace{3mm}
{\Large{\bf{Data Driven Policy Experience}}}
\begin{itemize}\itemsep1pt \parskip2pt \parsep0pt
	\item {\large {\bf Chair of William and Mary's physics department Climate Steering Committee (CSC)} }
	\begin{itemize}\itemsep1pt \parskip2pt \parsep0pt
 %\singlespacing{
		\item $\bullet$ While chairing the CSC I developed and implemented evidence based policy recommendations for the department's administration with the goal of improving the department's climate. 
		\item $\bullet$ Performed data gathering and fact finding through multiple methods including:
			\begin{itemize}
				\item i. Scholarly literature on climate issues and best practices were studied.
				\item ii. Experts on climate were consulted to learn successful techniques to address climate issues.
				\item iii. Surveys distributed to department constituencies such as faculty and students to identify issues.
				\item iv. Department-wide hearings were held to hear concerns and suggestions for how to improve policy.
			\end{itemize}
		\item $\bullet$ Developed policy recommendations based on the scholarly research collected and expert advise to address the issues discovered.
		\item $\bullet$ Communicated policy recommendations to the administration and explained how those recommendations were reached and how they could best be implemented.
		\item $\bullet$ Key policy accomplishments of the CSC under my leadership include:
			\begin{itemize}
				\item i. Created and implemented a Diversity Plan for the department.
				\item ii. Established a new departmental committee to improve colloquia quality.
				\item iii. Ratified a Code of Conduct and Statement of Values with department support and approval.
				\item iv. Informed the public of the committee's work by publishing updates on policies being developed.
				\item v. Implemented annual climate surveys so that the efficacy of new policies can be tracked objectively.
			\end{itemize}
	\end{itemize}
\end{itemize}

{\Large{\bf{Scientific Expertise}}}
\begin{itemize}\itemsep1pt \parskip2pt \parsep0pt
	\item {\large {\bf Nuclear Physics at Thomas Jefferson National Accelerator Facility (JLab)} }
	\begin{itemize}\itemsep1pt \parskip2pt \parsep0pt
 %\singlespacing{
 		\item $\bullet$ Built and designed particle detectors for upcoming experiments at JLab.
		\item $\bullet$ Created and tested data acquisition system for GRINCH Cherenkov detector.
		\item $\bullet$ Extracted nuclear form factors for $^3$He and $^3$H using elastic electron scattering.
		\item $\bullet$ Engaged in self-directed literature reviews to learn new scientific topics needed for my research. 
		\item $\bullet$ Participated in running numerous experiments at JLab including DVCS, GMp, Ar(e,e'p), and the Tritium suite of experiments. Publications are currently in progress for these experiments. 
	\end{itemize}
	\item {\large {\bf Data Analysis} }
	\begin{itemize}\itemsep1pt \parskip2pt \parsep0pt
 %\singlespacing{
		\item $\bullet$ Analyzed large data sets to extract physics quantities including:
			\begin{itemize}
				\item i. Detector efficiencies.
				\item ii. Nuclear form factors.
				\item iii. Good electron samples.
			\end{itemize}
		\item $\bullet$ Developed software to monitor the electronic dead time of the JLab data acquisition system in real-time to ensure quality data is collected.
		\item $\bullet$ Gathered world data for $^3$He and $^3$H form factors, and performed a new set of global fits to the data incorporating new data points and extracting charge radii.
	\end{itemize}
		\item {\large {\bf Computational Physics} }
	\begin{itemize}\itemsep1pt \parskip2pt \parsep0pt
 %\singlespacing{
 		\item $\bullet$ Created computer simulations of particle detectors using GEANT4.
		\item $\bullet$ Used Monte Carlo techniques to model physical processes and confirm experimental outcomes.
		\item $\bullet$ Familiar with many different programming languages including C++, ROOT, Python, Fortran, Mathematica, VHDL, Tcl, and Bash scripting.
	\end{itemize}
\end{itemize}

{\Large{\bf{Communication Skills}}}
\begin{itemize}\itemsep1pt \parskip2pt \parsep0pt
 %\singlespacing{
 	\item $\bullet$ Skilled at taking large amounts of complex data and presenting it in an easily digestible form.
 	\item $\bullet$ Collaborated with scientists from institutions around the world to perform and analyze experiments.
 	\item $\bullet$ Coordinated with designers and engineers to build equipment to required specifications for experiments at JLab.
	\item $\bullet$ Regularly gave presentations and progress reports at collaboration meetings and presented results at conferences.
	\item $\bullet$ Developed and wrote the procedures for performing numerous tasks during experimental runs at JLab so that shift workers unfamiliar with the experiments could perform the work properly. 
	\item $\bullet$ Worked with industry representatives to communicate our experimental needs and develop an appropriate product within budgetary constraints. This required explaining technical issues in a manner that those without subject matter expertise could understand.
\end{itemize}

{\Large{\bf{Leadership}}}
\begin{itemize}\itemsep1pt \parskip2pt \parsep0pt
	\item $\bullet$ Mentored junior graduate students and undergraduates as they learned to work at a national laboratory.
	\item $\bullet$ Chaired the College of William and Mary's physics department Climate Steering Committee and coordinated all of the committee's work. 
	\item $\bullet$ Took initiative to investigate software systems with anomalous readings and discovered numerous errors that were corrected improving data quality.
	\item $\bullet$ Routinely organized and lead teams of graduate students in accomplishing experimental goals such as data analysis, hardware maintenance, and communicating research results to others.
\end{itemize}

%\section*{Education}
{\Large{\bf{Education}}}
\begin{itemize}\itemsep1pt \parskip2pt \parsep0pt
 %\singlespacing{	
  \item B.S. Physics, B.S. Mathematics, B.S. Astronomy, Drake University, 2012.
  \item M.S. Physics, College of William and Mary, 2014.
  \item Ph.D. Physics, College of William and Mary, expected August 2018.
\end{itemize}

%\clearpage

%\section*{Publications}
{\Large{\bf{Publications}}}
%\subsection*{Papers}
\begin{itemize}\itemsep1pt \parskip2pt \parsep0pt
	\item {\large{\bf{Papers}}}
	\begin{enumerate}\itemsep1pt \parskip2pt \parsep0pt
		\item  Scott Barcus \textit{et al.} Near-unity nuclear polarization with an open-source 129Xe hyperpolarizer for NMR and MRI PNAS 2013 ; published ahead of print August 14, 2013, doi:10.1073/pnas.1306586110
	\end{enumerate}
\end{itemize}
%\subsection*{Talks and Posters}
\begin{itemize}\itemsep1pt \parskip2pt \parsep0pt
	\item {\large{\bf{Talks}}}
	\begin{enumerate}\itemsep1pt \parskip2pt \parsep0pt
		\setcounter{enumi}{1}
		%\begin{itemize}\itemsep1pt \parskip0pt \parsep0pt
		\item \textit{``Dynamic Mass of Self-Interacting Fermions''} Poster. Annual Meeting of the Iowa Academy of Sciences, Mason City, IA, April 2012.

		\item \textit{"9125B ET Photomultiplier Tubes with a Wavelength Shifting Paint for a Gas Cherenkov Counter"} Talk presented to the 2015 April Meeting of the APS. Balitmore, MD, April 11 2015. 
	\end{enumerate}
\end{itemize}
%\end{itemize}

\bigskip
% Footer
\begin{center}
  \begin{footnotesize}
    Last updated: \today \\
    \href{\footerlink}{\texttt{\footerlink}}
  \end{footnotesize}
\end{center}
}
\end{document}